\documentclass[11pt]{article}
\usepackage[margin=1in]{geometry}
\usepackage{fancyhdr}
\usepackage{listings}
\usepackage{hyperref}
\usepackage{titlesec}

\titleformat{\section}{\large\bfseries}{\thesection}{1em}{}

\pagestyle{fancy}
\fancyhf{}
\rhead{Software Analysis and Design III}
\lhead{CS335}
\rfoot{Page \thepage}

\title{\textbf{Assignment: Hash Table with AVL Tree Chaining}}
\author{}
\date{}

\begin{document}

\maketitle

\section*{Objective}
The goal of this assignment is to help you understand how to implement a hash table with separate chaining using \textbf{AVL trees} instead of linked lists. You will then use the hash table in a real-world application involving a \textbf{library book catalog system}.

\section*{Part 1: Data Structure Implementation}

\subsection*{Description}
Implement a hash table in C++ where:
\begin{itemize}
  \item Each bucket uses an \textbf{AVL tree} for collision resolution.
  \item Keys are \texttt{std::string} (e.g., book titles or ISBNs).
  \item The AVL tree supports efficient operations in $\mathcal{O}(\log n)$ time per bucket.
\end{itemize}

\subsection*{Requirements}
\begin{enumerate}
  \item \textbf{AVL Tree}:
    \begin{itemize}
      \item Self-balancing binary search tree.
      \item Implement insertion, search and deletion.
      \item Each node stores a key-value pair (\texttt{std::string}, \texttt{std::string}).
    \end{itemize}
  \item \textbf{Hash Table}:
    \begin{itemize}
      \item Fixed-size array of AVL tree pointers.
      \item String hash function (e.g., polynomial rolling hash or \texttt{std::hash}).
      \item Public operations:
        \begin{itemize}
          \item \texttt{void insert(const std::string\& key, const std::string\& value);}
          \item \texttt{std::string* search(const std::string\& key);}
          \item \texttt{void delete(const std::string\& key);}
          \item \texttt{void display();} — show table contents by bucket.
        \end{itemize}
    \end{itemize}
\end{enumerate}

\section*{Part 2: Application — Library Book Catalog}

\subsection*{Scenario}
You are building a system to store and retrieve book information by title using your hash table.

\subsection*{Requirements}
\begin{itemize}
  \item Each book has:
    \begin{itemize}
      \item Title (string) — used as the key.
      \item Author (string)
      \item Genre (string)
    \end{itemize}
  \item Store book data in the hash table using the title as the key.
  \item Value format: a combined string of author and genre (e.g., \texttt{F. Scott Fitzgerald - Fiction}).
  \item Load books from a file named \texttt{books.txt} formatted as:
\begin{lstlisting}[basicstyle=\ttfamily]
The Great Gatsby,F. Scott Fitzgerald,Fiction
A Brief History of Time,Stephen Hawking,Science
...
\end{lstlisting}
  \item Implement a simple console menu:
\begin{lstlisting}[basicstyle=\ttfamily]
1. Load books from file
2. Search for a book by title
3. Display hash table
4. Exit
\end{lstlisting}
\end{itemize}

\section*{Deliverables}
\begin{itemize}
  \item \texttt{AVLTree.h / AVLTree.cpp}
  \item \texttt{HashTable.h / HashTable.cpp}
  \item \texttt{main.cpp}
  \item \texttt{books.txt} (sample file with at least 10 entries)
  \item \textbf{Report (1 page)} including:
    \begin{itemize}
      \item Description of the AVL tree.
      \item Explanation of integration with the hash table.
      \item Time and space complexity analysis.
    \end{itemize}
\end{itemize}

\section*{Hints}
\begin{itemize}
  \item Use AVL tree balancing (LL, RR, LR, RL rotations) to maintain height.
  \item Balance factor = height(left) - height(right).
  \item Use \texttt{std::hash<std::string>} for simple hashing.
\end{itemize}

\section*{Grading Criteria}

\begin{tabular}{|l|c|}
\hline
\textbf{Component} & \textbf{Points} \\
\hline
AVL Tree Implementation & 25 \\
Hash Table Implementation & 20 \\
File I/O and Integration & 15 \\
Search Functionality & 10 \\
Display Function & 10 \\
Code Quality \& Documentation & 10 \\
Report & 10 \\
\hline
\end{tabular}


\end{document}
